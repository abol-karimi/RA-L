\subsection{The Planner}
\label{sec:voronoiplanner}

Our planner is \emph{reactive} in the sense that it does not remember its past inputs or outputs.
%
The input is simply a 2D point-cloud from lidar, and the output is a 2D \emph{waypoint} passed to the controller.
%
We assume that the environment is polygonal.
%
The planner calculates a Voronoi diagram corresponding to the point-cloud, then it chooses a point on the Voronoi diagram as the waypoint.

The first step is to calculate a Voronoi diagram.
%
Since the environment is polygonal, the planner converts the 2D point cloud to a set of line segments based on co-linearity and connectivity thresholds.
%
\emph{Co-linearity threshold} determines the minimum angle between consecutive line segments of a polyline.
%
\emph{Connectivity threshold} determines the minimum distance between two co-linear line segments.
%
Representing a set of points by a line segment simplifies the computation and representation of the Voronoi diagram.
%
Since the input to the Voronoi computation is a set of line segments, the Voronoi edges are either linear or parabolic arcs.
%
After computing the Voronoi diagram, we approximate each parabolic edge by a polyline using a \emph{deviation threshold}.
%
The \emph{deviation threshold} determines the maximum distance of the points on a parabolic arc from the approximating polyline.
%
This linear approximation simplifies the planner and its formal analysis.
%
A Voronoi diagram calculated by the planner is called a \emph{local} Voronoi diagram since it is computed for the point-cloud visible from lidar.
%
This is in contrast to the \emph{global} Voronoi diagram where the diagram is computed with respect to the whole polygonal environment (which is not available to the car).

The next step is to choose a waypoint based on the local Voronoi diagram.
%
We choose a point \emph{on} the Voronoi diagram to try to stay as far as possible from the track walls (i.e. to be as safe as possible).
%
If the waypoint is too close or too far from the vehicle, the controller may make the car steer too sharply or slowly.
%
We choose among the points at a fixed distance from the center of the rear axle.
%
This distance is called the \emph{lookahead distance} and the corresponding circle is called the \emph{lookahead circle}.
%
The lookahead circle may intersect the Voronoi diagram in more than one point, so we need to choose among them.
%
Since the goal is to make the car progress towards finishing the track, the intersection point further along the heading direction of the car is selected as the waypoint.
