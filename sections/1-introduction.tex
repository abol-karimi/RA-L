\section{Introduction}

%--- Why racing?
\emph{Racing} provides relatively simple environments and requirements for development, verification and validation of automated driving algorithms, e.g. path-planning, path-tracking, obstacle avoidance, object detection and tracking, etc.
%
Betz et al. \cite{Betz.2022} survey several autonomous racing competitions that have emerged over the past few years.
%
For example, the F1/10 competition uses 1/10 scale autonomous cars, the Indy Autonomous Challenge uses full-scale race cars, and the Generalized RAcing Intelligence Competition (GRAIC) is a simulation-only race with challenging dynamic-obstacle scenarios.
%
A \emph{racing task} may be simply finishing several laps, while remaining within the track boundaries, as in F1/10 time trials.
%
Furthermore, the task may involve avoiding static and dynamic obstacles on the track, as in GRAIC or head-to-head racing in F1/10.
%
A minimal requirement for such racing tasks is \emph{safety} and \emph{progress}.
%
Here \emph{safety} means avoiding obstacles and remaining within the track boundaries, while \emph{progress} means finishing one or more laps.
%
Furthermore, the \emph{performance} requirement is to make progress faster than other competitors.
%
In this paper we focus on the safety and progress requirements.


%--- The minimalistic approad to design
One approach to study motion planning is \emph{minimalism} \cite{Tovar.2009}.
%
That is, to study the minimum information and computing resources required to perform the planning task.
%
For example, a \emph{map-less} \cite{Tai.2017} planner is not given a map, neither it builds a map.
%
A \emph{localization-less} planner does not know the location of the vehicle relative to its goal, in contrast to using GPS or WiFi localization \cite{Tai.2017}.
%
An \emph{oblivious} \cite{Flocchini.2012} planner does not remember any information from the previous perceive-plan-control cycle.
%
A challenge in designing such minimalistic planners is to ensure their consistency between consecutive cycles.
%
We developed a \emph{map-less} \emph{oblivious} Voronoi-based motion planner which won the GRAIC 2021 competition.
%
A version of that planner that is furthermore \emph{localization-less} was developed for an F1/10 car and tested both in simulation and on the physical car, and verified for a bicycle model of a car and several polygonal models of race circuits.
%
This shows the potential of Voronoi diagrams for minimalistic motion planning with guarantees.


%--- Planner overview
Our algorithm for obstacle-free racing is a perceive-plan-control loop.
%
The perception gives the track boundaries within a sensing radius of the car as a set of line segments.
%
The planner computes the Voronoi diagram of these track boundaries, then outputs a line segment as the path to follow.
%
This line segment is either a Voronoi edge or a linear approximation of a parabolic Voronoi edge.
%
The controller tracks the planned path using a pure-pursuit path-tracker.
%
Note that a nonholonomic car cannot perfectly track the planned path.



%--- Main contribution
The main contribution of this paper is the safety and progress analysis for a \emph{map-less} \emph{localization-less} \emph{oblivious} motion planner.
%
One of the challenges with verification of the planner is the complexity of the perceive-plan-control dynamical system due to the complexity of perception (computing visible track boundaries), and lack of localization data to bypass that complexity.
%
The complexities of the planner, controller and the controlled plant (bicycle model) further complicate the closed-loop system function.
%
The core novelty of our guarantee is proving a \emph{local consistency} property of Voronoi diagrams, which ensures that the Voronoi diagram of the track boundaries that are within a local area of the car is equal to the Voronoi diagram of all track boundaries.
%
Therefore, the dynamical system can be defined with respect to the Voronoi diagram of the whole track which does not change and can be computed offline, and enables bypassing the complexity of the perception function.
%
We believe such properties of Voronoi diagrams and their applications deserve more research.
%
The contributions of this paper are as follows:
\begin{enumerate}
\item
Development of a map-less localization-less oblivious motion planner.
\item
Proving a local consistency property of Voronoi diagrams.
\item
Modeling a car together with its planner, controller and the racing track as a hybrid automata.
\item
Verification of the safety and progress of the planner on several tracks using off-the-shelf tools for reachability analysis of hybrid automata.
\end{enumerate}



%--- Modeling with hybrid automata
We model the perceive-plan-control closed-loop with a hybrid automata.
%
Each discrete mode of the automata corresponds to the closed-loop when following a fixed line segment.
%
The discrete mode changes when the planner returns a different line segment.
%
The state of the system in a discrete mode is the \emph{pose} of the car i.e. its 2D position and 1D orientation.
%
The steering angle of the car is determined from the pose, the planned path, and the pure-pursuit formula.
%
This results in a continuous non-linear dynamics in each discrete mode of the automata.
%
The \emph{initial state} of the system is an initial discrete mode and an initial pose.
%
Uncertainty in the initial pose is modeled as an \emph{initial set} of poses.


%-- Reachable-set computation
A hybrid automata can be simulated from an initial state until a finite time horizon, resulting the trajectory of the car.
%
An initial set results in a set of trajectories.
%
The states encountered in the trajectories comprise the \emph{reachable set} of the system over the time horizon.
%
The reachable set then can be checked for safety and progress.
%
For example, if all the trajectories pass the finish line then progress is established.
%
If the reachable set does not grow after a time horizon, the set is a \emph{fixed-point} of the closed-loop.
%
A fixed-point is the same as the reachable set over an infinite time horizon.
%
Using a fixed-point we can prove that the system is safe over an infinite time horizon, or that it finishes an infinite number of laps.
