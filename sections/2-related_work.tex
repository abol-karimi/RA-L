\section{Related work}
\label{sec:relwork}
Reducing the reliance on maps has several motivations.
%
For example, \cite{Ort.2019} notes that 
obtaining and maintaining a detailed map is expensive, especially that roads change due to seasonal vegetation, road and building construction, road maintenance, etc.
%
They develop a system called MapLite that can navigate roads with only a sparse topometric map, namely the OpenStreetMap of the environment.
%
Another application noted in \cite{Vasudevan.2021}, \cite{Deruyttere.2019} is when the navigation task is not defined by a source and destination on a map, but by verbal instructions from a user such as ``go ahead for about 200m until you hit a small intersection, then turn left and continue along the street before you see a yellow building on your right.''


Since our focus is on guarantees, we only review the related literature on planning and control that give formal guarantees of safety and progress.


In \cite{Ivanov.2020}, an end-to-end neural-network controller is developed for an F1/10 race car.
%
Similar to our work, the controller is map-less;
%
It takes the raw LiDAR measurements as input and outputs the steering commands.
%
An advantage of our system is its interpretability due to the decomposition of the task to planning and control, where the plan is a path computed from the Voronoi diagram of the track boundaries.
%
They perform safety verification with respect to a right-angle turn between two straight hallways.
%
In contrast, our verification is for several polygonal circuit tracks with variable turn angles and track widths.
%
Furthermore, we verify progress for any number of laps around the circuit.


