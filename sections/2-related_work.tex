\section{Related work}
\label{sec:relwork}
In \cite{Ivanov.2020, Ivanov-verifying.2020}, an end-to-end neural-network controller is developed for an F1/10 race car.
%
Similar to our work, the controller is map-less and localization-less;
%
It takes the raw LiDAR measurements as input and outputs the steering commands.
%
An advantage of our system is its interpretability due to the decomposition of the task to path-planning and path-tracking.
%
They perform safety and progress verification with respect to a single right-angle turn between two straight hallways.
%
In contrast, our verification is for several polygonal circuit tracks with variable turn angles and track widths.
%
Furthermore, we verify progress for any number of laps around the circuit.
%
Similar to our work, they use \emph{reachable-set computation} for the verification.



In \cite{Bohrer.2019} a Simplex architecture \cite{Seto.1998} is used to develop controllers with formal guarantees.
%
In this approach a \emph{monitor} is composed with an untrusted controller to check and correct its commands if they may lead to unsafe states.
%
They provide both safety and progress guarantees and the complexity of the tracks they use compare to ours.
%
However, they assume that a safe waypoint plan is provided, and the guarantees are for the waypoint-following controller only.
%
That is, the design of the planner is neither discussed nor verified.
%
Furthermore, their proofs are not fully automated and need human ingenuity to help in generating loop invariants and progress functions.



Reducing the reliance on maps has several motivations.
%
For example, \cite{Ort.2019} notes that 
obtaining and maintaining a detailed map is expensive, especially that roads change due to seasonal vegetation, road and building construction, road maintenance, etc.
%
They develop a system called MapLite that can navigate roads with only a sparse topometric map, namely the OpenStreetMap of the environment.
%
Another application noted in \cite{Vasudevan.2021}, \cite{Deruyttere.2019} is when the navigation task is not defined by a source and a destination on a map, but by verbal instructions from a user such as ``go ahead for about 200m until you hit a small intersection, then turn left and continue along the street before you see a yellow building on your right.''


For a broader context on testing, verification, and validation of autonomous vehicles see \cite{Rajabli.2020, Zhang.2020} for systematic literature reviews.



