\section{Appendix}

\subsection{Reachable set accuracy}
\label{subsec:appendix-reach-set-accuracy}
Suppose the track is defined in a stationary $(x, y)$ frame, $\theta$ is the vehicle's orientation and 
$\ell$ is the look ahead distance. 
%
% Fig.~\ref{fig:sim1_flow} represents the car's trajectory starting from a particular initial configuration, as computed in Flow*. 
As a part of experimentation, we perform the reachable set computation in Flow* and CORA by modeling a single turn on the track as the hybrid automata. 
%
The initial set is given as intervals over state variables i.e., $x \in [0.0, 0.5]$, $y \in [0.9, 1.1]$ and $\theta \in [-0.5, -0.5]$. 
%
The reachable sets computed in both tools are shown in figures~\ref{fig:reachSet1_flow} and ~\ref{fig:reachSet1_cora}. 
%
The divergent behavior of the reachable set in Flow* is seemingly due to error compounded over time because of coarse approximation. 
%
The figures also illustrate that the vehicle while turning swings to some extent before merging back on to the track.

\begin{figure}
    \centering
    \subfigure[Flow*]{\label{fig:reachSet1_flow}\includegraphics[width=3.5in]{Figures/reachSet_1_flow.png}}
    \subfigure[CORA]{\label{fig:reachSet1_cora}\includegraphics[width=\linewidth]{Figures/reachSet_1_cora.png}}
    \caption{Reachable sets computed in Flow* and CORA with time step 0.02 sec and time bound 15 sec for a set of initial states. The vehicle follows a vertical path downwards before making  a left turn.}
\label{fig:reachSets}
\end{figure}

% \begin{figure}
% \centering
% % \subfigure[]{%
% % \includegraphics[width=2.8in]{Figures/flow_pure_pursuit_4.png}%
% % \label{fig:sim1_flow}%
% % }
% \subfigure[]{%
% \includegraphics[width=\linewidth]{Figures/reachSet_1_flow.png}%
% \label{fig:reachSet1_flow}%
% }
% \subfigure[]{%
% \includegraphics[width=\linewidth]{Figures/reachSet_1_cora.png}%
% \label{fig:reachSet1_cora}%
% }
% \caption{Simulation results in Flow* and CORA with time step 0.02 sec and time bound 15 sec. The vehicle follows a vertical path downwards before making  a left turn. (a) shows that the Flow* trajectory diverges for single initial configuration. Subsequent plots (b) and (c) represent the reachable sets computed in Flow* and CORA, respectively, for a set of initial states.}
% \label{fig:reachSets}
% \end{figure}

% \subsection{Fixed points and reachable sets for various tracks}

\begin{figure*}
    \centering
    \subfigure[Track 1-X]{\label{fig:track1-loc3-x}\includegraphics[width=3.8in]{Figures/tracks/traci1-loc3-x.jpg}}
    \subfigure[Track 1-Y]{\label{fig:track1-loc3-y}\includegraphics[width=3.8in]{Figures/tracks/track1-loc3-y.jpg}}
    \subfigure[Track 1-$\theta$]{\label{fig:track1-loc3-theta}\includegraphics[width=3.8in]{Figures/tracks/track1-loc3-theta.jpg}}
    \caption{\textbf{Fixed point illustration using state variables in Track-1}}
\label{fig:track1-fixed-point-loc3}
\end{figure*}

% \begin{figure}
% \centering
% \includegraphics[width=3.5in]{Figures/tracks/track2-voronoi-cora.jpg}%
% \caption{Verification on a general track. The fixed point is achieved after completing the first lap.}
% \label{fig:eval_track2}
% \end{figure}

\begin{figure*}
    \centering
    \subfigure[Track 2]{\label{fig:track2-voronoi-cora}\includegraphics[width=3.5in]{Figures/tracks/track2-voronoi-cora.jpg}}
    \subfigure[Track 3]{\label{fig:track3-voronoi-cora}\includegraphics[width=3.5in]{Figures/tracks/track3-voronoi-cora.jpg}}
    \subfigure[Track 4]{\label{fig:track4-voronoi-cora}\includegraphics[width=3.5in]{Figures/tracks/track4-voronoi-cora.jpg}}
    \subfigure[Track 5]{\label{fig:track5-voronoi-cora}\includegraphics[width=3.5in]{Figures/tracks/track5-voronoi-cora.jpg}}
    \caption{Safety verification of the plan in multiple other tracks. The track-wise initial sets are $\Theta_{2} = [[-0.25, 0.25][-0.25, 0.25][-0.15,0.15]]$, $\Theta_{3} = [[-0.2, 0.2][-0.2, 0.2][-0.2,0.2]]$, $\Theta_{4} = [[-0.12, 0.12][-0.12, 0.12][-0.12,0.12]]$, and $\Theta_{5} = [[-0.12, 0.12][-0.12, 0.12][-0.12,0.12]]$}
\label{fig:eval_track2-track3-track4-track5}
\end{figure*}


% \begin{figure}
% \centering
% \subfigure[]{%
% \includegraphics[width=3.5in]{Figures/tracks/track2-voronoi-cora.jpg}%
% \label{fig:track2_voronoi-cora}%
% }\qquad
% \subfigure[]{%
% \includegraphics[width=3.5in]{Figures/tracks/track3-voronoi-cora.jpg}%
% \label{fig:track3_voronoi-cora}%
% }
% \caption{Verification of tracks 2 and 3.}
% \label{fig:eval_track2-track3}
% \end{figure}


