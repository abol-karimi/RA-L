\section{Evaluation}
In this section, we evaluate our fixed-point based reachable set computation on five different tracks with different characteristics. 
%
For each track given as the sequence of global Voronoi diagram edges, a new hybrid automaton model for the evolution of vehicle along this track is generated. Typically, a hybrid automata has as many discrete modes as the number of Voronoi edges. 
%
The dynamics in each mode of the  automaton is the closed loop behavior of the vehicle dynamics and the way point as described in Section~\ref{sec:model}.
%
Then, we employ a non-linear hybrid systems verification tool to computed the fixed point of the reachable set for each track.
%is obtained with respect to a given initial set. 
%
The computation of a fixed point in each track underscores both safety and progress of the vehicle.
%
In addition, we tested the planning and control algorithm both in simulation and open source F1Tenth platform.\footnote{\url{https://tinyurl.com/ry5xhza}}
% Videos of the experiments are available on the web.\footnote{\url{https://tinyurl.com/ry5xhza}} 
% Once the closed loop behavior of the vehicle dynamics is obtained as discussed in the Section~\ref{sec:model}, 
% we employ a non-linear hybrid system verification tool to compute the reachable set for each track.

\subimport{}{5-1-verification.tex}
